%-----------------------------------------------------------------------------
%
%               Template for sigplanconf LaTeX Class
%
% Name:         sigplanconf-template.tex
%
% Purpose:      A template for sigplanconf.cls, which is a LaTeX 2e class
%               file for SIGPLAN conference proceedings.
%
% Guide:        Refer to "Author's Guide to the ACM SIGPLAN Class,"
%               sigplanconf-guide.pdf
%
% Author:       Paul C. Anagnostopoulos
%               Windfall Software
%               978 371-2316
%               paul@windfall.com
%
% Created:      15 February 2005
%
%-----------------------------------------------------------------------------


\documentclass{jfp1}

% The following \documentclass options may be useful:

% preprint      Remove this option only once the paper is in final form.
% 10pt          To set in 10-point type instead of 9-point.
% 11pt          To set in 11-point type instead of 9-point.
% authoryear    To obtain author/year citation style instead of numeric.

\usepackage{xcolor}
\usepackage{xspace}
%\usepackage{amsmath}
%\usepackage{amsthm}
\usepackage{doubleequals}
\usepackage{comment}
\usepackage{url}
\usepackage{graphicx}
\usepackage[export]{adjustbox}
\graphicspath{ {images/} }
\usepackage{dirtytalk}

%% for comments
\input{Preamble}

\newtheorem{theorem}{Theorem}[section]
\newtheorem{lemma}[theorem]{Lemma}

\def\commentbegin{\quad\{\ }
\def\commentend{\}}

\newcommand\SM[1]{\textcolor{red!70!yellow}{[SM: #1]}}
\newcommand\bruno[1]{\textcolor{green!70!black}{[BO: #1]}}
\newcommand\SH[1]{\textcolor{blue!70!gray}{[SH: #1]}}

\newcommand\emma[1]{\textcolor{magenta}{[EM: #1]}}

\newcommand\name{{\bf DSL}\xspace}
\newcommand\delete[1]{}

\begin{document}

\special{papersize=8.5in,11in}
\setlength{\pdfpageheight}{\paperheight}
\setlength{\pdfpagewidth}{\paperwidth}

%\conferenceinfo{CONF 'yy}{Month d--d, 20yy, City, ST, Country} 
%\copyrightyear{20yy} 
%\copyrightdata{978-1-nnnn-nnnn-n/yy/mm} 
%\doi{nnnnnnn.nnnnnnn}

% Uncomment one of the following two, if you are not going for the 
% traditional copyright transfer agreement.

%\exclusivelicense                % ACM gets exclusive license to publish, 
                                  % you retain copyright

%\permissiontopublish             % ACM gets nonexclusive license to publish
                                  % (paid open-access papers, 
                                  % short abstracts)

%\titlebanner{banner above paper title}        % These are ignored unless
%\preprintfooter{short description of paper}   % 'preprint' option specified.

\title{Compositionality and Modularity Together at Last!}
\subtitle{(Functional Pearl)}

\author[B. Peng and B. C. d. S. Oliveira]
        {BOYA PENG and BRUNO C. d. S. OLIVEIRA\\
         The University of Hong Kong, Hong Kong\\
         \email{bruno@cs.hku.hk}}

%\authorinfo{Draft}
%           {}
%           {}

\maketitle

\begin{abstract}
Compositionality is a desirable characteristic for the semantics of
DSLs. In functional programming folds can be used to enforce
compositionality. However folds make certain compositional
interpretations hard to express modularly. This pearl presents a
simple technique that allows non-trivial, but compositional
interpretations of DSLs to be expressed in a fully modular way
using folds.
\end{abstract}

%\category{CR-number}{subcategory}{third-level}

% general terms are not compulsory anymore, 
% you may leave them out
%\terms
%term1, term2

%\keywords
%keyword1, keyword2

\input{sections/Introduction}

\input{sections/Technique}

\input{sections/Scans}

\input{sections/FAlg}

%\input{sections/Overview}

\input{sections/MultipleAlg}

\input{sections/DependentAlg}

\input{sections/Context} 

%\input{sections/Extensibility}

\input{sections/TypeClassAlg}

%\input{sections/RecordAlg}

\input{sections/Grammar}

%\input{sections/RelatedWork}

\input{sections/Conclusion}

\begin{comment}
\appendix
\section{Appendix Title}

This is the text of the appendix, if you need one.

\acks

Acknowledgments, if needed.
\end{comment}

% We recommend abbrvnat bibliography style.

\bibliographystyle{abbrvnat}

% The bibliography should be embedded for final submission.

%\bibliography{mrm}

\begin{thebibliography}{}
%\softraggedright

\bibitem{sheard04}
  Sheard, Tim and Pasalic, Emir.
  Two-level types and parameterized modules.
  \emph{Journal of Functional Programming}.
  14(5):547-587, September 2014.

\bibitem{gibbons14}
  Gibbons, Jeremy and Wu, Nicolas.
  Folding Domain-Specific Languages: Deep and Shallow Embeddings.
  \emph{International Conference on Functional Programming}.
  2014.

\bibitem{blelloch90}  
  Blelloch, Guy E..
  Prefix sums and their applications
  \emph{Technical Report}
  1990

\bibitem{fokkinga90}
  Fokkinga, Maarten M.
  Tupling and mutumorphisms.
  1990.

\bibitem{oliveira15}
  Oliveira, B.C.d.S., Mu, Shin-Cheng and You, Shu-Hung.
  Modular Reifiable Matching: A List-of-Functors Approach to Two-Level Types.
  February 2015.

\bibitem{bahr15}
  Bahr, Patrick and Axelsson, Emil.
  Generalising Tree Traversals to DAGs.
  \emph{Proceedings of the 2015 Workshop on Partial Evaluation and Program Manipulation},
  pages 27-38.
  2015

\bibitem{oliveira13}
  Oliveira, B.C.d.S., Storm, Tijs van der, Loh, Alex and Cook, William R.
  Feature-Oriented Programming with Object Algebra
  \emph{ECOOP'13 Proceedings of the 27th European conference on Object-Oriented Programming},
  pages 27-51.
  2013

\bibitem{gibbons98}
  Gibbons, Jeremy and Jones, Geraint.
  The under-appreciated unfold.
  \emph{International Conference on Functional Programming},
  pages 273-279.
  Baltimore, Maryland, September 1998

\end{thebibliography}



\end{document}

%                       Revision History
%                       -------- -------
%  Date         Person  Ver.    Change
%  ----         ------  ----    ------

%  2013.06.29   TU      0.1--4  comments on permission/copyright notices

